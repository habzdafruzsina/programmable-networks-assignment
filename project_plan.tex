\documentclass{article}
\usepackage[utf8]{inputenc}
\usepackage[english]{babel}
\usepackage[margin=2in]{geometry}

\sloppy
\geometry{margin=1.5in}

\title{In-network feature collection of H264 or H265 streams - Project plan}
\author{Fruzsina Habzda, Balázs Racskó, Ákos András Kajtár}
\date{April 2024}

\begin{document}

\maketitle

\section{Project Overview}

The aim of our project is to collect data from H264 and H265 streams at ...

This involves parsing I and P frames, collecting per stream features, and transmitting telemetry packets to a collector server for analysis.

\section{Goals}

Finding, identifying, and parsing I and P frames of H264 or H265 streams.

Collecting per stream features like i-frame rate, p-frame rate, i-frame sizes, p-frame sizes, and inter-frame gaps.

Implementing a telemetry packet generator to transmit collected features to a telemetry collector server.

Developing a telemetry collector server to receive and analyze telemetry packets.

\section{Detailed Plan}

To achieve this, we have to:

\begin{itemize}
\item research and understand H264 ans H265 streams, I and P frames, and real-time streaming in whole
\item write parsers for I and P frames
\item implement the extraction of frame features
\item implement logic to calculate some of the collectable data (e.g. sizes and gaps)
\item develop a system to collect the given data (I'm not yet sure about the sequence of these steps)
\item create  a telemetry collector server (design the storage and communication method as well)
\item implement a telemetry packet generator
\item integrate the system (packet generator, parser, collector server)
\item test the system
\item prepare a presentation of our solution
\item present the project
\end{itemize}


\section{Due dates}

\begin{itemize}
\item 22 Apr Project plan
\item 13 May (?) Presentation
\item 6 Jul Finish
\end{itemize}

\end{document}